%%%%%%%% ICML 2024 EXAMPLE LATEX SUBMISSION FILE %%%%%%%%%%%%%%%%%

\documentclass{article}

% Recommended, but optional, packages for figures and better typesetting:
\usepackage{microtype}
\usepackage{graphicx}
\usepackage{subfigure}
\usepackage{booktabs} % for professional tables

% hyperref makes hyperlinks in the resulting PDF.
% If your build breaks (sometimes temporarily if a hyperlink spans a page)
% please comment out the following usepackage line and replace
% \usepackage{icml2024} with \usepackage[nohyperref]{icml2024} above.
\usepackage{hyperref}


% Attempt to make hyperref and algorithmic work together better:
\newcommand{\theHalgorithm}{\arabic{algorithm}}

% Use the following line for the initial blind version submitted for review:
\usepackage{icml2024}

% If accepted, instead use the following line for the camera-ready submission:
% \usepackage[accepted]{icml2024}

% For theorems and such
\usepackage{amsmath}
\usepackage{amssymb}
\usepackage{mathtools}
\usepackage{amsthm}
\usepackage{bm}


% if you use cleveref..
\usepackage[capitalize,noabbrev]{cleveref}

%%%%%%%%%%%%%%%%%%%%%%%%%%%%%%%%
% THEOREMS
%%%%%%%%%%%%%%%%%%%%%%%%%%%%%%%%
\theoremstyle{plain}
\newtheorem{theorem}{Theorem}[section]
\newtheorem{proposition}[theorem]{Proposition}
\newtheorem{lemma}[theorem]{Lemma}
\newtheorem{corollary}[theorem]{Corollary}
\theoremstyle{definition}
\newtheorem{definition}[theorem]{Definition}
\newtheorem{assumption}[theorem]{Assumption}
\theoremstyle{remark}
\newtheorem{remark}[theorem]{Remark}

% Todonotes is useful during development; simply uncomment the next line
%    and comment out the line below the next line to turn off comments
%\usepackage[disable,textsize=tiny]{todonotes}
% \usepackage[textsize=tiny]{todonotes}


% The \icmltitle you define below is probably too long as a header.
% Therefore, a short form for the running title is supplied here:
\icmltitlerunning{Submission and Formatting Instructions for ICML 2024}

\DeclareMathOperator{\ilr}{ilr}
\DeclareMathOperator{\tr}{tr}

\begin{document}

\twocolumn[
\icmltitle{Explaining probabilistic prediction on the simplex with Shapley compositions}

% It is OKAY to include author information, even for blind
% submissions: the style file will automatically remove it for you
% unless you've provided the [accepted] option to the icml2024
% package.

% List of affiliations: The first argument should be a (short)
% identifier you will use later to specify author affiliations
% Academic affiliations should list Department, University, City, Region, Country
% Industry affiliations should list Company, City, Region, Country

% You can specify symbols, otherwise they are numbered in order.
% Ideally, you should not use this facility. Affiliations will be numbered
% in order of appearance and this is the preferred way.
\icmlsetsymbol{equal}{*}

\begin{icmlauthorlist}
\icmlauthor{Firstname1 Lastname1}{equal,yyy}
\icmlauthor{Firstname2 Lastname2}{equal,yyy,comp}
\icmlauthor{Firstname3 Lastname3}{comp}
\icmlauthor{Firstname4 Lastname4}{sch}
\icmlauthor{Firstname5 Lastname5}{yyy}
\icmlauthor{Firstname6 Lastname6}{sch,yyy,comp}
\icmlauthor{Firstname7 Lastname7}{comp}
%\icmlauthor{}{sch}
\icmlauthor{Firstname8 Lastname8}{sch}
\icmlauthor{Firstname8 Lastname8}{yyy,comp}
%\icmlauthor{}{sch}
%\icmlauthor{}{sch}
\end{icmlauthorlist}

\icmlaffiliation{yyy}{Department of XXX, University of YYY, Location, Country}
\icmlaffiliation{comp}{Company Name, Location, Country}
\icmlaffiliation{sch}{School of ZZZ, Institute of WWW, Location, Country}

\icmlcorrespondingauthor{Firstname1 Lastname1}{first1.last1@xxx.edu}
\icmlcorrespondingauthor{Firstname2 Lastname2}{first2.last2@www.uk}

% You may provide any keywords that you
% find helpful for describing your paper; these are used to populate
% the "keywords" metadata in the PDF but will not be shown in the document
\icmlkeywords{Machine Learning, ICML}

\vskip 0.3in
]

% this must go after the closing bracket ] following \twocolumn[ ...

% This command actually creates the footnote in the first column
% listing the affiliations and the copyright notice.
% The command takes one argument, which is text to display at the start of the footnote.
% The \icmlEqualContribution command is standard text for equal contribution.
% Remove it (just {}) if you do not need this facility.

%\printAffiliationsAndNotice{}  % leave blank if no need to mention equal contribution
\printAffiliationsAndNotice{\icmlEqualContribution} % otherwise use the standard text.

\begin{abstract}
  The concept of Shapley value has been widely use for measuring the contribution of each feature on a machine learning model's prediction. However, this has been designed for one-dimensional function's codomain. For multiclass probabilistic classifier, where the output is a discrete probability distribution over the set of more than two possible classes, the output lives on a multidimensional simplex. In this case, people have been applying the concept of Shapley value on each output dimension one-by-one, in an implicit one-vs-rest setting, ignoring the compositional nature of the output distribution where the relative information between probabilities matter. Using the Aitchison geometry of the simplex, coming from the field of compositional data analysis, this paper present a first initiative for a multidimensional extention of the concept of Shapley value, named Shapley composition, for explaining probabilistic predictions on the simplex in machine learning.
\end{abstract}

\section{Introduction}

Modern machine learning approaches like the one based on deep learning are often regarded as black-boxes making them not reliable for real-life application where the machine learnbing prediction has to be understood. These last years, the number of contribution to make models more explainable has therefore increased in the machine learning literature. One way to better understand a prediction would be to measure the contribution of each input features on the computation of the model output. The concept of Shapley value is now widely used for this purpose \cite{vstrumbelj2014explaining,datta2016} especially since the release of the SHAP toolkit \cite{NIPS2017_7062}\footnote{\url{https://github.com/shap/shap}}. The Shapley value came from cooperative game theory...

explain shapley in game theory,

How it is applied to ML,

Limitation,

...

\subsection{Contributions}

...

\section{The Shapley value in machine learning}
\label{sec:shapley}

In this section, we recall the theoretical formulation of the Shapley value for measuring the contribution of each feature on a machine learning prediction.

Let $f:\mathcal{X}\to\mathbb{R}$ be a learned model one want to \emph{locally} explain where $f(\bm{x})$ is the prediction on the instance $\bm{x}\in\mathcal{X}$. Let $\text{Pr}$ be the probability distribution over $\mathcal{X}$ of the data\footnote{In practice, this is usually unknown but the expectation will be replaced by empirical samplings.}. Let $S\subseteq \mathcal{I}=\{1,2,\dots d\}$, where $d$ is the number of features that composes an instance $\bm{x}\in\mathcal{X}$, be a subset of indices. $\bm{x}_S$ refers to an instance $\bm{x}$ restricted to the features indicated by the indices in $S$.

When an instance $\bm{x}$ is observed, the expected value of the prediction is simply $\mathbb{E}[f(\bm{x}) \mid \bm{x}] = f(\bm{x})$. However, when only $\bm{x}_S$ is given, i.e. part of the features, there is uncertainty about the other features and we therefore compute the expected prediction given $\bm{x}_S$: $\mathbb{E}_{\text{Pr}}[f(\bm{x}) \mid \bm{x}_S] = \int_{\bm{x} \in \mathcal{X}}f(\bm{x})\text{Pr}(\bm{x} \mid \bm{x}_S)d\bm{x}$. The contribution of the feature indexed by $i \notin S$ in the prediction $f(\bm{x})$ given the known features indexed by $S$ is given by:
\begin{equation}
  \label{eq:contrib}
  c_{f,\bm{x},\text{Pr}}(i,\bm{X}_S) = v_{f,\bm{x},\text{Pr}}(\bm{X}_{S\cup\{i\}}) - v_{f,\bm{x},\text{Pr}}(\bm{X}_S),
\end{equation}
where $v$ is known as the value function:
\begin{equation}
  \label{eq:valuefunction}
  \begin{aligned}
    v_{f,\bm{x},\text{Pr}}: 2^{\mathcal{I}} &\to \mathbb{R},\\
    S &\mapsto \mathbb{E}_\text{Pr}[f(\bm{x})\mid \bm{x}_S],
  \end{aligned}
\end{equation}
where $2^{\mathcal{X}}$ is the set of all subsets of $I$. This measure the contribution of the $i$th features with a particular \emph{coalition} of features indexed by $S$. The whole contribution of the $i$th feature is computing by averaging this quantity over all possible coalitions as follow:
\begin{equation}
  \phi_{f,\bm{x},\text{Pr}}(i) = \frac{1}{d!} \sum_{\pi}c_{f,\bm{x},\text{Pr}}(i,\pi^{<i}_{\bm{X}}),
\end{equation}
where $\pi$ is a permutation of the set $I$ of indexes and $\pi^{<i}_{\bm{X}}$ is the features of $\bm{X}$ coming before the $i$th feature in the ordering given by $\pi$. For better clarity, the subscript $_{f,\bm{x},\text{Pr}}$ wil be dropt from the equations.

This quantity is known as the Shapley value for the $i$th feature. It comes from cooperative game theory and is known to be only quantity respecting a set of desired axiomatic properties \cite{shapley1953value}. It is linear as a function of the model ($\alpha, \beta \in \mathbb{R}$): $\phi_{\alpha f +\beta g}(i) = \alpha \phi_f(i) + \beta \phi_g(i)$, and the ``centered'' learned model is additively separable with respect to the Shapley values: $(\bm{x})-\mathbb{E}_{\text{Pr}}[f(\bm{X})] = \sum_{i=1}^{d} \phi_f(i)$, which is known as the \emph{efficiency} property.

Like originally developed in game theory, the Shapley value is designed for one-dimensinal codomain of the function $f$. For explaining machine learning models which output multidimensional discrete probability distribution, like in multiclass classification, people have been explaining each output dimension one-by-one, applying a logit transformation to the probabilities, resulting in a one-vs-rest comparison. However, this ignores the relative information between each probability and the compositional nature of probability distributions. Indeed, the probabilistic output of a classifier lives on a multidimensional simplex. The latter is the sample space of data refered as \emph{compositional data} we briefly review in the next section.

\section{Compositional data}

Compositional data carries relative information. Each element of a composition \emph{describes a part of some whole} \cite{pawlowskymodeling} like vectors of proportions, concentrations, and discrete probability distributions. A $N$-part composition is a vector of $N$ non-zero positive real numbers that sum to a constant $k$. Each element of the vector is a part of the \emph{whole} $k$. The sample space of compositional data is known as the simplex: $\mathcal{S}^N = \left\{ \bm{x} = [x_1, x_2,\dots x_{N}]^T \in \mathbb{R}^{*N}_{+} \big| \sum_{i=1}^{N} x_i = k \right\}$. In a composition, only the relative information between parts matters and John Aitchison introduced the use of log-ratios of components to handle this \cite{aitchison1982}. He defined several operations on the simplex which leads to what is called the \emph{Aitchison geometry of the simplex}.

\subsection{The Aitchison geometry of the simplex}
John Aitchison defined an internal operation called \emph{perturbation}, an external one called \emph{powering} and an inner product \cite{aitchison2001}:
\begin{itemize}
\item a \emph{perturbation}:
  $\bm{x}\oplus \bm{y} = \mathcal{C}\left([x_1y_1,\dots x_{N}y_{N}]\right)$ seen as an addition between two compositions $\bm{x},\bm{y}\in \mathcal{S}^N$,
\item a \emph{powering}:
  $\alpha \odot \bm{x} = \mathcal{C}\left([x_{1}^{\alpha},\dots x_{N}^{\alpha}]\right)$ seen as a multiplication by a scalar $\alpha \in \mathbb{R}$,
\item an inner product:\\
  $\displaystyle \langle \bm{x},\bm{y} \rangle_a = \frac{1}{2N}\sum_{i=1}^{N} \sum_{j=1}^{N} \log \frac{x_i}{x_j}\log \frac{y_i}{y_j}$.
\end{itemize}
$\mathcal{C}(\cdot)$ is the closure operator. Since only the relative information matter, scaling factors are irrelevant and a composition $\bm{x}$ is equivalent to $\lambda \bm{x} = [\lambda x_1,\lambda x_2,\dots\lambda x_N]$ for all $\lambda>0$. This equivalence is materialized by the closure operator defined for $k>0$ as: $\mathcal{C}\left(\bm{x} \right) = \left[ \frac{k x_1}{\lVert \bm{x} \rVert_1}, \frac{k x_2}{\lVert \bm{x} \rVert_1} ,\dots \frac{k x_N}{\lVert \bm{x} \rVert_1} \right]^T$, where $\bm{x} \in \mathbb{R}_+^{*N}$ and $\displaystyle \lVert \bm{x} \rVert_1 = \sum_{i=1}^N \lvert x_i \rvert$.%Therefore, any vector of positive real numbers can be projected onto the simplex using the closure.

This give to the simplex a $(N-1)$-dimensional Euclidean vector space structure called \emph{Aitchison geometry of the simplex}. In this paper, since we are interested in classifiers' outputs as discrete probability distributions, we restrict ourselves to the \emph{probability simplex} where $k=1$.

\subsection{The isometric log-ratio transformation}

An $(N-1)$-dimensional orthonormal basis of the simplex, refered as an \emph{Aitchison} orthonormal basis, can be built. The projection of a composition (a discrete probability distribution in our case) into this basis defined an isometric isomorphism between $\mathcal{S}^N$ and $\mathbb{R}^{N-1}$. This is known as an Isometric-Log-Ratio (ILR) transformation \cite{egozcue2003isometric} and allows to express the compositions into a Cartesian coordinates system preserving the metric of the Aitchison geometry. Within this real space, the permutation, the powering and the Aitchison inner product defined above are respectively the standard addition, multiplication by a scalar and inner product.

Given a composition $\bm{p} = \left[ p_1,\dots p_N \right]^T \in \mathcal{S}^N$ we express is ILR transformation as $\tilde{\bm{p}} = \ilr \left( \bm{p} \right) = \left[ \tilde{p}_1,\dots \tilde{p}_{N-1} \right]^T \in \mathbb{R}^{N-1}$. The $i$th element $\tilde{p}_i$ of $\tilde{\bm{p}}$ is be obtained as: $\tilde{p}_i = \langle \bm{p}, \bm{e}^{(i)} \rangle_a$ where set $\{\bm{e}^{(i)} \in \mathcal{S}^N, i=1,\dots N-1\}$ forms an \emph{Aitchison} orthonormal basis of the simplex. The choice of the basis will be discussed latter in Section \ref{sec:explain}.


\section{Shapley on the simplex}

In Section \ref{sec:shapley} we have briefly presented the standard formulation of Shapley value designed for one-dimensional prediction. In this Section, we will see how the Aitchison geometry can be used to extend this concept to the multidimensional simplex for explaning probabilistic predictions.


Let $\bm{f}:\mathcal{X}\to\mathcal{S}^N$ be a learned model, like a $N$-classes probabilistic classifier for instance, which outputs a probabilistic prediction on the $(N-1)$-dimensional probability simplex $S^N$. To properly consider the relative information between the probabilities, The outputs of the model must be treated as compositional data using the operators and metric defined by the Aitchison geometry of the simplex. We therefore rewrite the contribution and the value function of Equations \ref{eq:contrib} and \ref{eq:valuefunction} as follow:
\begin{equation}
  \bm{c}_{\bm{f},\bm{x},\text{Pr}}(i,\bm{X}_S) = \bm{v}_{\bm{f},\bm{x},\text{Pr}}(\bm{X}_{S\cup\{i\}}) \ominus \bm{v}_{\bm{f},\bm{x},\text{Pr}}(\bm{X}_S),
\end{equation}
where $\bm{a}\ominus\bm{b}$ is the perturbation $\bm{a} \oplus \left( (-1)\odot \bm{b}\right)$ which correspond to a substraction between compositions $\bm{a}$ and $\bm{b}$, and where:
\begin{equation}
  \label{eq:valuefunctionsimplex}
  \begin{aligned}
    \bm{v}_{\bm{f},\bm{x},\text{Pr}}: 2^{\mathcal{X}} &\to \mathcal{S}^N,\\
    \bm{X}_S &\mapsto \mathbb{E}^{\mathcal{A}}_\text{Pr}[\bm{f}(\bm{x})\mid \bm{x}_S].
  \end{aligned}
\end{equation}
The $\mathcal{A}$ in superscript highlight the fact that the expectation is done with respect to the Aitchison measure, rather than the Lebesgue measure, which can simply be computed as \cite{pawlowskymodeling}: $\mathbb{E}^{\mathcal{A}}[\bm{Y}] = \ilr^{-1}\left( \mathbb{E} \left[ \ilr\left( \bm{Y} \right) \right] \right)$,
where $\mathbb{E}^{\mathcal{A}}$ refers to the expectation with respect to the Aitchison measure while $\mathbb{E}$ refers to the expectation with respect to the Lebesgue measure.

The Shapley quantity expressing the contribution of the $i$th feature on a prediction can simply be expressed on the simplex as the \emph{Shapley composition} $\bm{\phi}(i)$ given by:
\begin{equation}
  \bm{\phi}_{\bm{f},\bm{x},\text{Pr}}(i) = \frac{1}{d!} \odot \underset{\pi}{\bigoplus}\bm{c}_{\bm{f},\bm{x},\text{Pr}}(i,\pi^{<i}_{\bm{X}}).
\end{equation}
It can be shown (in Appendix \ref{app:properties}) that the linearity and the efficiency properties naturally hold for the Shapley composition:
\begin{equation}
  \begin{aligned}
    \bm{\phi}_{\alpha \odot \bm{f}(\bm{x}) \oplus \beta \odot \bm{g}(\bm{x})}(i) &= \alpha \odot \bm{\phi}_{\bm{f}}(i) \oplus \beta\odot \bm{\phi}_{\bm{g}}(i),\\
    \underset{i=1}{\overset{d}\bigoplus} \bm{\phi}_{\bm{f}}(i) &= \bm{f}(\bm{x}) \ominus \mathbb{E}^{\mathcal{A}}_{\text{Pr}}[\bm{f}(\bm{X})].
  \end{aligned}
\end{equation}
This can be seen as a multidimensional extension of the Shapley value framework on the simplex. Here, the Shapley quantity is not a scalar anymore, this is a composition living on the probability simplex. In the next section, we will see in more details how this can be used to explain the contribution of the features on a multidimensional probabilistic prediction.

\section{Explaining probabilistic prediction with Shapley compositions}
\label{sec:explain}


Given a prediction $\bm{f}(\bm{x})$, the Shapley composition $\bm{\phi}_{\bm{f},\bm{x},\text{Pr}}(i)$ describes the contribution of the $i$th feature on the prediction. The efficiency property shows how the probability distribution moves from the base value, i.e. the expected prediction regardless of the current input, to the prediction $\bm{f}(\bm{x})$. In the standard Shapley formulation recalled in Section \ref{sec:shapley}, the prediction is one-dimensional such that the Shapley quantity is a scalar. In application where there are more than two possible classes, the prediction is multidimensional such that the Shapley quantity is too. Both lives on the same space: the probability simplex. In this section, we discuss how the set of Shapley compositions can be analysed to better understand the contribution and influence of each features on the prediction.

\subsection{Angles, norms and projections}

\subsection{Visualization}

\subsubsection{Three classes}

\subsubsection{Four classes}

\subsubsection{More classes}

\subsubsection{Histograms}

\subsection{About our implementation}

In this work, the estimation algorithm we used to compute the Shapley compositions is an adaptation of Algorithm 2 in \cite{vstrumbelj2014explaining}. Since the resulting Shapley compositions are approximations, the efficiency property does not necessarily hold without an adjustement. Each Shapley compositions are therefore adjusted following a similar method as in the SamplingExplainer of the SHAP toolkit \cite{NIPS2017_7062}\footnote{\url{https://github.com/shap/shap/blob/master/shap/explainers/_sampling.py}}. See Appendix \ref{app:algo} for more details.

\section{Discussion and conclusion}


% In the unusual situation where you want a paper to appear in the
% references without citing it in the main text, use \nocite
\nocite{langley00}

\bibliography{biblio}
\bibliographystyle{icml2024}


%%%%%%%%%%%%%%%%%%%%%%%%%%%%%%%%%%%%%%%%%%%%%%%%%%%%%%%%%%%%%%%%%%%%%%%%%%%%%%%
%%%%%%%%%%%%%%%%%%%%%%%%%%%%%%%%%%%%%%%%%%%%%%%%%%%%%%%%%%%%%%%%%%%%%%%%%%%%%%%
% APPENDIX
%%%%%%%%%%%%%%%%%%%%%%%%%%%%%%%%%%%%%%%%%%%%%%%%%%%%%%%%%%%%%%%%%%%%%%%%%%%%%%%
%%%%%%%%%%%%%%%%%%%%%%%%%%%%%%%%%%%%%%%%%%%%%%%%%%%%%%%%%%%%%%%%%%%%%%%%%%%%%%%
\newpage
\appendix
\onecolumn

\section{Linearity and efficiency of the Shapley composition}
\label{app:properties}
In this section, we show the linearity of the Shapley composition with respect to the model prediction, and the efficiency property.

\subsection{Linearity}

The Shapley composition is linear, within the Aitchison geometry of the simplex, with respect to linear combination of models' predictions.
\begin{proof}
  Let's consider the linear combination of predictions $\bm{h}(\bm{x}) = \alpha \odot \bm{f}(\bm{x}) \oplus \beta \odot \bm{g}(\bm{x})$. we want to check if:
  \begin{equation}
    \label{eq:linearsimplex}
    \bm{\phi}_{\bm{h}}(i) = \alpha \odot \bm{\phi}_{\bm{f}}(i) \oplus \beta\odot \bm{\phi}_{\bm{g}}(i).
  \end{equation}
  We have:
  \begin{equation}
    \begin{aligned}
      \mathbb{E}^{\mathcal{A}}_{\text{Pr}}[\bm{h}(\bm{x})\mid \bm{x}_S] &= \ilr^{-1} \left( \mathbb{E}_{\text{Pr}} [\ilr \left(  \alpha \odot \bm{f}(\bm{x}) \oplus \beta \odot \bm{g}(\bm{x}) \right) \mid \bm{x}_S] \right),\\
                                                                        &= \ilr^{-1} \left( \mathbb{E}_{\text{Pr}} [\alpha \ilr \left( \bm{f}(\bm{x})\right) + \beta \ilr \left( \bm{g}(\bm{x}) \right) \mid \bm{x}_S] \right),\\
                                                                        &= \ilr^{-1} \left( \alpha \mathbb{E}_{\text{Pr}} [ \ilr \left( \bm{f}(\bm{x})\right) \mid \bm{x}_S] + \beta \mathbb{E}_{\text{Pr}} [ \ilr \left( \bm{g}(\bm{x}) \right) \mid \bm{x}_S] \right),\\
                                                                        &= \alpha \odot \ilr^{-1} \left( \mathbb{E}_{\text{Pr}} [ \ilr \left( \bm{f}(\bm{x})\right) \mid \bm{x}_S] \right) \oplus \beta \odot \ilr^{-1} \left( \mathbb{E}_{\text{Pr}} [ \ilr \left( \bm{g}(\bm{x}) \right) \mid \bm{x}_S] \right),\\
                                                                        &= \alpha \odot \mathbb{E}^{\mathcal{A}}_{\text{Pr}}[\bm{f}(\bm{x})\mid \bm{x}_S] \oplus \beta \odot \mathbb{E}^{\mathcal{A}}_{\text{Pr}}[\bm{g}(\bm{x})\mid \bm{x}_S].
    \end{aligned}
  \end{equation}
  Therefore, $\bm{v}_{\bm{h},\bm{x},\text{Pr}}(\bm{X}_S) = \alpha \odot \bm{v}_{\bm{f},\bm{x},\text{Pr}}(\bm{X}_S) \oplus \beta \odot \bm{v}_{\bm{g},\bm{x},\text{Pr}}(\bm{X}_S)$, meaning that $\bm{v}$ is linear with respect to the learned function or model. The linearity of the contribution $\bm{c}$ naturally follows:
  \begin{equation}
    \begin{aligned}
      &\bm{c}_{\bm{h},\bm{x},\text{Pr}}(i,\bm{X}_S) = \bm{v}_{\bm{h},\bm{x},\text{Pr}}(\bm{X}_{S\cup\{i\}}) \ominus \bm{v}_{\bm{h},\bm{x},\text{Pr}}(\bm{X}_S),\\
      &~~~~~~~= \left( \alpha \odot \bm{v}_{\bm{f},\bm{x},\text{Pr}}(\bm{X}_{S\cup\{i\}}) \oplus \beta \odot \bm{v}_{\bm{g},\bm{x},\text{Pr}}(\bm{X}_{S\cup\{i\}}) \right) \ominus \left( \alpha \odot \bm{v}_{\bm{f},\bm{x},\text{Pr}}(\bm{X}_S) \oplus \beta \odot \bm{v}_{\bm{g},\bm{x},\text{Pr}}(\bm{X}_S) \right),\\
      &~~~~~~~= \alpha \odot \bm{v}_{\bm{f},\bm{x},\text{Pr}}(\bm{X}_{S\cup\{i\}}) \oplus \beta \odot \bm{v}_{\bm{g},\bm{x},\text{Pr}}(\bm{X}_{S\cup\{i\}}) \ominus\alpha \odot \bm{v}_{\bm{f},\bm{x},\text{Pr}}(\bm{X}_S) \ominus \beta \odot \bm{v}_{\bm{g},\bm{x},\text{Pr}}(\bm{X}_S),\\
      &~~~~~~~= \alpha \odot \left( \bm{v}_{\bm{f},\bm{x},\text{Pr}}(\bm{X}_{S\cup\{i\}}) \ominus \bm{v}_{\bm{f},\bm{x},\text{Pr}}(\bm{X}_S) \right) \oplus \beta \odot \left( \bm{v}_{\bm{g},\bm{x},\text{Pr}}(\bm{X}_{S\cup\{i\}}) \ominus \bm{v}_{\bm{g},\bm{x},\text{Pr}}(\bm{X}_S)\right),\\
      &~~~~~~~= \alpha \odot \bm{c}_{\bm{f},\bm{x},\text{Pr}}(i,\bm{X}_S) \oplus \beta \odot \bm{c}_{\bm{g},\bm{x},\text{Pr}}(i,\bm{X}_S).
    \end{aligned}
  \end{equation}
  And the linearity of the Shap composition:
  \begin{equation}
    \begin{aligned}
      \bm{\phi}_{\bm{h}}(i) &= \frac{1}{d!}  \underset{\pi}{\bigoplus}\bm{c}_{\bm{h},\bm{x},\text{Pr}}(i,\pi^{<i}_{\bm{X}}),\\
                            &= \frac{1}{d!}  \underset{\pi}{\bigoplus}\left( \alpha \odot \bm{c}_{\bm{f},\bm{x},\text{Pr}}(i,\bm{X}_S) \oplus \beta \odot \bm{c}_{\bm{g},\bm{x},\text{Pr}}(i,\bm{X}_S) \right),\\
                            &= \alpha \odot \left( \frac{1}{d!}  \underset{\pi}{\bigoplus} \bm{c}_{\bm{f},\bm{x},\text{Pr}}(i,\bm{X}_S) \right) \oplus \beta \odot \left( \frac{1}{d!} \underset{\pi}{\bigoplus}    \bm{c}_{\bm{g},\bm{x},\text{Pr}}(i,\bm{X}_S) \right),\\
                            &= \alpha \odot \bm{\phi}_{\bm{f}}(i) \oplus \beta\odot \bm{\phi}_{\bm{g}}(i).
    \end{aligned}
  \end{equation}
\end{proof}

\newpage
\subsection{Efficiency}
The efficiency property naturally holds for Shapley compositions within the Aitchison geometry.
\begin{proof}
  \begin{equation}
    \begin{aligned}
      \underset{i=1}{\overset{d}\bigoplus} \bm{\phi}_{\bm{f}}(i) & = \underset{i=1}{\overset{d}\bigoplus}\left( \frac{1}{d!} \odot \underset{\pi}{\bigoplus} \bm{c}(i,\pi_{\bm{X}}^{<i})\right),\\
                                                                 &=  \frac{1}{d!} \odot \underset{i=1}{\overset{d}\bigoplus}\left( \underset{\pi}{\bigoplus} \left( \bm{v}(\pi_{\bm{X}}^{<i+1}) \ominus \bm{v}(\pi_{\bm{X}}^{<i}) \right) \right),\\
                                                                 &=  \frac{1}{d!} \odot \underset{i=1}{\overset{d}\bigoplus}\left( \underbrace{\left(\underset{\pi}{\bigoplus} \bm{v}(\pi_{\bm{X}}^{<i+1}) \right)}_{\bm{A}_{i+1}} \ominus \underbrace{\left( \underset{\pi}{\bigoplus} \bm{v}(\pi_{\bm{X}}^{<i}) \right)}_{\bm{A}_{i}} \right),\\
                                                                 &=  \frac{1}{d!} \odot \underset{i=1}{\overset{d}\bigoplus}\left( \bm{A}_{i+1} \ominus \bm{A}_{i} \right),\\
                                                                 &=  \frac{1}{d!} \odot \left( \bm{A}_{d+1} \ominus \bm{A}_{1} \right), \text{~since we have a telescoping perturbation,}\\
                                                                 &=  \frac{1}{d!} \odot \left( \left( \underset{\pi}{\bigoplus} \bm{v}(\pi_{\bm{X}}^{<d+1}) \right) \ominus \left( \underset{\pi}{\bigoplus} \bm{v}(\pi_{\bm{X}}^{<1}) \right) \right),\\
                                                                 &=  \frac{1}{d!} \odot \left( \left( \underset{\pi}{\bigoplus} \bm{v}(\bm{X}) \right) \ominus \left( \underset{\pi}{\bigoplus} \bm{v}(\bm{X}_{\emptyset}) \right) \right),\\
                                                                 &= \bm{v}(\bm{X}) \ominus \bm{v}(\bm{X}_{\emptyset}), \text{~since $d!$ is the number of different permutation,}\\
                                                                 &= \bm{f}(\bm{x}) \ominus \mathbb{E}^{\mathcal{A}}_{\text{Pr}}[\bm{f}(\bm{X})].
    \end{aligned}
  \end{equation}
\end{proof}


\section{Class compositions}

A $k$-class compositions is defined as an unit norm composition going straight to the direction of the $k$th class. This is a discrete probability distribution with maximum probability for the $k$th class ...\\
...\\
...

\section{Estimation of the Shapley compositions}
\label{app:algo}
In this work, we used an adaptation of Algorithm 2 in \cite{vstrumbelj2014explaining} we present in this section.

Let $d$ be the number of features. We want to optimally distribute the $m_{\text{max}}$ drawn samples over the $d$ features. Let $\hat{\bm{\phi}_i}$ the estimation of the Shapley composition for the $i$th feature. We want to minimize the sum of squared errors or distances $\displaystyle\sum_{i=1}^{d} \| \hat{\bm{\phi}_i} \ominus \bm{\phi}_i \|_a^2$. Since $\hat{\bm{\phi}_i}$ is a (Aitchison) sample mean we have: $\tilde{\hat{\bm{\phi}_i}} \approx \mathcal{N}(\tilde{\bm{\phi}}_i, \frac{1}{m_i}\bm{\Sigma}^{(i)})$ and $\tilde{\hat{\bm{\phi}_i}} - \tilde{\bm{\phi}}_i \approx \mathcal{N}(\bm{0}, \frac{1}{m_i}\bm{\Sigma}^{(i)})$ where the tilde refers to the ILR transformation.
Let $\bm{\Delta_i} = \tilde{\hat{\bm{\phi}_i}} - \tilde{\bm{\phi}}_i$ and $Z_i = \| \hat{\bm{\phi}_i} \ominus \bm{\phi}_i \|_a = \| \tilde{\hat{\bm{\phi}_i}} - \tilde{\bm{\phi}}_i \|_2 = \| \bm{\Delta}_i \|_2$. The expectation of the sum of squared errors is:
\begin{equation}
  \begin{aligned}
    \mathbb{E}\left[ \sum_{i=1}^{d} Z_i^2 \right] &=   \sum_{i=1}^{d} \mathbb{E}\left[ Z_i^2 \right],\\
                                                  & =\sum_{i=1}^{d} \mathbb{E}\left[ \sum_{j=1}^{d-1}\Delta_{ij}^2 \right],\\
                                                  & =\sum_{i=1}^{d} \sum_{j=1}^{d-1} \mathbb{E}\left[\Delta_{ij}^2 \right],\\
                                                  & =\sum_{i=1}^{d} \sum_{j=1}^{d-1} \frac{1}{m_i}\Sigma_{jj}^{(i)}, \text{ since } \Delta_{ij}\approx \mathcal{Z}(0, \frac{1}{m_i}\Sigma_{jj}^{(i)}),\\
    &= \sum_{i=1}^d \frac{1}{m_i} \tr \bm{\Sigma}^{(i)}.
  \end{aligned}
\end{equation}

When a sample is drawn, the feature for which the sample will be used for improving the Shapley composition estimation is chosen to maximize $\frac{\tr \bm{\Sigma}^{(i)}}{m_i} - \frac{\tr \bm{\Sigma}^{(i)}}{m_i+1}$. Like in \cite{vstrumbelj2014explaining}, this is summarized in Algorithm \ref{alg:2}.
\begin{algorithm}[tb]
   \caption{Adaptation of the Algorithm 1 in \cite{vstrumbelj2014explaining} for approximating the Shapley composition of the $i$th feature, with model $\bm{f}$, instance $\bm{x}\in\mathcal{X}$ and $m$ drawn samples.}
   \label{alg:1}
\begin{algorithmic}
   \STATE Initialize $\bm{\phi_i}\leftarrow \ilr^{-1}(\bm{0})$
   \FOR{$1$ {\bfseries to} $m$}
   \STATE Randomly select a permutation $\pi$ of the set of indexes $\mathcal{I}$,
   \STATE Randomly select a sample $\bm{w}\in\mathcal{X}$,
   \STATE Construct two instances:
   \begin{itemize}
     \item $\bm{b}_1$: which takes the values from $\bm{x}$ for the $i$th feature and the features indexed before $i$ in the order given by $\pi$, and takes the values from $\bm{w}$ otherwise,
     \item $\bm{b}_2$: which takes the values from $\bm{x}$ the features indexed before $i$ in the order given by $\pi$, and takes the values from $\bm{w}$ otherwise.
     \end{itemize}
   \STATE $\bm{\phi}_i \leftarrow \bm{\phi}_i \oplus \bm{f}(\bm{b}_1) \ominus \bm{f}(\bm{b}_2) $
   \ENDFOR
   \STATE $\bm{\phi}_i \leftarrow \frac{\bm{\phi}_i}{m}$
\end{algorithmic}
\end{algorithm}

\begin{algorithm}[tb]
   \caption{Adaptation of the Algorithm 2 in \cite{vstrumbelj2014explaining} for approximating all the Shapley compositions by optimally distributing a maximum number of samples $m_{\text{max}}$ over the $d$ features, with model $\bm{f}$, instance $\bm{x}\in\mathcal{X}$ and $m_{\text{min}}$ the minimum number of samples each feature estimation.}
   \label{alg:2}
   \begin{algorithmic}
     \STATE Initialization: 
\end{algorithmic}
\end{algorithm}

%%%%%%%%%%%%%%%%%%%%%%%%%%%%%%%%%%%%%%%%%%%%%%%%%%%%%%%%%%%%%%%%%%%%%%%%%%%%%%% 
%%%%%%%%%%%%%%%%%%%%%%%%%%%%%%%%%%%%%%%%%%%%%%%%%%%%%%%%%%%%%%%%%%%%%%%%%%%%%%%


\end{document}


% This document was modified from the file originally made available by
% Pat Langley and Andrea Danyluk for ICML-2K. This version was created
% by Iain Murray in 2018, and modified by Alexandre Bouchard in
% 2019 and 2021 and by Csaba Szepesvari, Gang Niu and Sivan Sabato in 2022.
% Modified again in 2023 and 2024 by Sivan Sabato and Jonathan Scarlett.
% Previous contributors include Dan Roy, Lise Getoor and Tobias
% Scheffer, which was slightly modified from the 2010 version by
% Thorsten Joachims & Johannes Fuernkranz, slightly modified from the
% 2009 version by Kiri Wagstaff and Sam Roweis's 2008 version, which is
% slightly modified from Prasad Tadepalli's 2007 version which is a
% lightly changed version of the previous year's version by Andrew
% Moore, which was in turn edited from those of Kristian Kersting and
% Codrina Lauth. Alex Smola contributed to the algorithmic style files.

%%% Local Variables:
%%% mode: latex
%%% TeX-master: t
%%% End:
